%% Generated by Sphinx.
\def\sphinxdocclass{report}
\documentclass[letterpaper,10pt,english]{sphinxmanual}
\ifdefined\pdfpxdimen
   \let\sphinxpxdimen\pdfpxdimen\else\newdimen\sphinxpxdimen
\fi \sphinxpxdimen=.75bp\relax
%% turn off hyperref patch of \index as sphinx.xdy xindy module takes care of
%% suitable \hyperpage mark-up, working around hyperref-xindy incompatibility
\PassOptionsToPackage{hyperindex=false}{hyperref}

\PassOptionsToPackage{warn}{textcomp}

\catcode`^^^^00a0\active\protected\def^^^^00a0{\leavevmode\nobreak\ }
\usepackage{cmap}
\usepackage{fontenc}
\usepackage{amsmath,amssymb,amstext}
\usepackage{polyglossia}
\setmainlanguage{english}


\setmainfont{TeX Gyre Termes}
\setsansfont{TeX Gyre Termes}
\setmonofont{DejaVu Sans Mono}

\usepackage[Bjornstrup]{fncychap}
\usepackage{sphinx}

\fvset{fontsize=\small}
\usepackage[vmargin=2.5cm, hmargin=3cm]{geometry}


% Include hyperref last.
\usepackage{hyperref}
% Fix anchor placement for figures with captions.
\usepackage{hypcap}% it must be loaded after hyperref.
% Set up styles of URL: it should be placed after hyperref.
\urlstyle{same}


\usepackage{sphinxmessages}
\setcounter{tocdepth}{5}
\setcounter{secnumdepth}{5}
\usepackage[titles]{tocloft}
\cftsetpnumwidth {1.25cm}\cftsetrmarg{1.5cm}
\setlength{\cftchapnumwidth}{0.75cm}
\setlength{\cftsecindent}{\cftchapnumwidth}
\setlength{\cftsecnumwidth}{1.25cm}

\title{DiNest}
\date{Aug 29, 2021}
\release{1.0}
\author{djanloo}
\newcommand{\sphinxlogo}{\vbox{}}
\renewcommand{\releasename}{Release}
\makeindex
\begin{document}

\pagestyle{empty}
\sphinxmaketitle
\pagestyle{plain}
\sphinxtableofcontents
\pagestyle{normal}
\phantomsection\label{\detokenize{index::doc}}



\chapter{Generated code description}
\label{\detokenize{autogen:generated-code-description}}\label{\detokenize{autogen::doc}}\begin{description}
\item[{Code description generated automatically from docstrings.}] \leavevmode\phantomsection\label{\detokenize{autogen:module-my_NS}}\index{module@\spxentry{module}!my\_NS@\spxentry{my\_NS}}\index{my\_NS@\spxentry{my\_NS}!module@\spxentry{module}}
Nested Sampling implementation
author: G.B.
\index{NS() (in module my\_NS)@\spxentry{NS()}\spxextra{in module my\_NS}}

\begin{fulllineitems}
\phantomsection\label{\detokenize{autogen:my_NS.NS}}\pysiglinewithargsret{\sphinxcode{\sphinxupquote{my\_NS.}}\sphinxbfcode{\sphinxupquote{NS}}}{\emph{log\_likelihood}, \emph{log\_prior}, \emph{bounds}, \emph{explore=<function metro\_gibbs>}, \emph{Nlive=100}, \emph{Npoints=numpy.inf}, \emph{X\_assessment='deterministic'}, \emph{shrink\_scale=False}, \emph{scale\_factor=None}, \emph{stop\_log\_relative\_increment=\sphinxhyphen{}10}, \emph{max\_attempts=1000000}, \emph{verbose\_search=False}, \emph{display\_progress=True}, \emph{chain\_L=100}}{}
NestedSampling algorithm
\begin{quote}\begin{description}
\item[{Parameters}] \leavevmode\begin{itemize}
\item {} 
\sphinxstyleliteralstrong{\sphinxupquote{log\_likelihood}} (\sphinxstyleliteralemphasis{\sphinxupquote{function}}) – function of a SINGLE ARGUMENT (parameters vector(s) Np \sphinxhyphen{} dimensional)
must be able to handle (M, Np) shaped arrays
returning a (M,) vector
Note: 1D case \sphinxhyphen{}> (M,) is wrong, use instead (M,1)

\item {} 
\sphinxstyleliteralstrong{\sphinxupquote{bounds}} (\sphinxstyleliteralemphasis{\sphinxupquote{tuple of lists}}) – 
Indicate the searching bounds of a Nd\sphinxhyphen{}dimensional cube giving the upper and lower bounds
len(bounds) must be 2
each element a Np\sphinxhyphen{}dimensional vector
Note: for 1D case the shape must be (*,1) instead of (*,)
\begin{quote}

use reshape(\sphinxhyphen{}1,1)
\end{quote}


\item {} 
\sphinxstyleliteralstrong{\sphinxupquote{Nlive}} (\sphinxstyleliteralemphasis{\sphinxupquote{int}}) – Number of live points

\item {} 
\sphinxstyleliteralstrong{\sphinxupquote{logX}} – X values used

\item {} 
\sphinxstyleliteralstrong{\sphinxupquote{logL}} – L values used

\item {} 
\sphinxstyleliteralstrong{\sphinxupquote{logZ}} – the estimated value of logZ (using square integration)

\item {} 
\sphinxstyleliteralstrong{\sphinxupquote{stats}} (\sphinxstyleliteralemphasis{\sphinxupquote{dict}}\sphinxstyleliteralemphasis{\sphinxupquote{(}}\sphinxstyleliteralemphasis{\sphinxupquote{array}}\sphinxstyleliteralemphasis{\sphinxupquote{,}}\sphinxstyleliteralemphasis{\sphinxupquote{array}}\sphinxstyleliteralemphasis{\sphinxupquote{,}}\sphinxstyleliteralemphasis{\sphinxupquote{array}}\sphinxstyleliteralemphasis{\sphinxupquote{, }}\sphinxstyleliteralemphasis{\sphinxupquote{array}}\sphinxstyleliteralemphasis{\sphinxupquote{, }}\sphinxstyleliteralemphasis{\sphinxupquote{bool}}\sphinxstyleliteralemphasis{\sphinxupquote{, }}\sphinxstyleliteralemphasis{\sphinxupquote{float}}\sphinxstyleliteralemphasis{\sphinxupquote{)}}) – ‘param points’ points in variable space
‘log\_posterior\_weights’:(np.ndarray)    log posterior weights,
‘attempts’:             (np.ndarray)    attempts history,
‘log\_weights’:          (np.ndarray)    logw,
‘clustering\_warning’:   (boolean)       label (work in progress) to detect clustering
‘dLogZ’:                (float)         error on LogZ

\end{itemize}

\item[{Returns}] \leavevmode
LogX, LogL, LogZ, stats

\end{description}\end{quote}

\end{fulllineitems}

\index{get\_Z\_error() (in module my\_NS)@\spxentry{get\_Z\_error()}\spxextra{in module my\_NS}}

\begin{fulllineitems}
\phantomsection\label{\detokenize{autogen:my_NS.get_Z_error}}\pysiglinewithargsret{\sphinxcode{\sphinxupquote{my\_NS.}}\sphinxbfcode{\sphinxupquote{get\_Z\_error}}}{\emph{\DUrole{n}{L}}, \emph{\DUrole{n}{Nlive}}, \emph{\DUrole{n}{N\_t\_vec}\DUrole{o}{=}\DUrole{default_value}{50}}}{}
Estimates the uncertainty of the value of Z given stochastic sample of X

\end{fulllineitems}

\index{logsumexp() (in module my\_NS)@\spxentry{logsumexp()}\spxextra{in module my\_NS}}

\begin{fulllineitems}
\phantomsection\label{\detokenize{autogen:my_NS.logsumexp}}\pysiglinewithargsret{\sphinxcode{\sphinxupquote{my\_NS.}}\sphinxbfcode{\sphinxupquote{logsumexp}}}{\emph{\DUrole{n}{arg}}}{}
Given a  vector {[}a1,a2,a3, … {]} returns log(e**a1 + e**a2 + …)

\end{fulllineitems}

\phantomsection\label{\detokenize{autogen:module-AIE_sampling}}\index{module@\spxentry{module}!AIE\_sampling@\spxentry{AIE\_sampling}}\index{AIE\_sampling@\spxentry{AIE\_sampling}!module@\spxentry{module}}
implementation of affine invariant ensemble MCMC
\index{AIE\_sampling() (in module AIE\_sampling)@\spxentry{AIE\_sampling()}\spxextra{in module AIE\_sampling}}

\begin{fulllineitems}
\phantomsection\label{\detokenize{autogen:AIE_sampling.AIE_sampling}}\pysiglinewithargsret{\sphinxcode{\sphinxupquote{AIE\_sampling.}}\sphinxbfcode{\sphinxupquote{AIE\_sampling}}}{\emph{\DUrole{n}{log\_f}}, \emph{\DUrole{n}{bounds}}, \emph{\DUrole{n}{space\_scale}}, \emph{\DUrole{n}{nwalkers}\DUrole{o}{=}\DUrole{default_value}{10}}, \emph{\DUrole{n}{nsteps}\DUrole{o}{=}\DUrole{default_value}{100}}, \emph{\DUrole{n}{verbose}\DUrole{o}{=}\DUrole{default_value}{False}}}{}
Affine invariant ensemble sampling algorithm
\begin{quote}\begin{description}
\item[{Parameters}] \leavevmode\begin{itemize}
\item {} 
\sphinxstyleliteralstrong{\sphinxupquote{log\_f}} (\sphinxstyleliteralemphasis{\sphinxupquote{function}}) – Function to be sampled

\item {} 
\sphinxstyleliteralstrong{\sphinxupquote{bounds}} (\sphinxstyleliteralemphasis{\sphinxupquote{2\sphinxhyphen{}tuple of arrays}}) – bounds of the space

\item {} 
\sphinxstyleliteralstrong{\sphinxupquote{space\_scale}} (\sphinxstyleliteralemphasis{\sphinxupquote{float}}) – stretch values are generated between space\_scale and (space\_scale)**(\sphinxhyphen{}1)

\item {} 
\sphinxstyleliteralstrong{\sphinxupquote{nwalkers}} (\sphinxstyleliteralemphasis{\sphinxupquote{int}}) – the number of walkers of the ensemble

\end{itemize}

\item[{Returns}] \leavevmode
the chain evolved organized as (time, walker, space\_variables)

\end{description}\end{quote}

\end{fulllineitems}

\index{dummyfunc() (in module AIE\_sampling)@\spxentry{dummyfunc()}\spxextra{in module AIE\_sampling}}

\begin{fulllineitems}
\phantomsection\label{\detokenize{autogen:AIE_sampling.dummyfunc}}\pysiglinewithargsret{\sphinxcode{\sphinxupquote{AIE\_sampling.}}\sphinxbfcode{\sphinxupquote{dummyfunc}}}{\emph{\DUrole{n}{u}}}{}
This is a dummy function.
\begin{quote}\begin{description}
\item[{Parameters}] \leavevmode\begin{itemize}
\item {} 
\sphinxstyleliteralstrong{\sphinxupquote{x}} – \sphinxcode{\sphinxupquote{bool}}
A true goal rosp

\item {} 
\sphinxstyleliteralstrong{\sphinxupquote{y}} – 
Variable with type unspecified

\begin{sphinxadmonition}{note}{Note:}
y can be anything
\end{sphinxadmonition}

\begin{sphinxadmonition}{note}{Note:}
napoleon\sphinxhyphen{}notes can be everywhere tho
\end{sphinxadmonition}


\end{itemize}

\item[{Returns}] \leavevmode

True if successful, False otherwise.

The return type is optional and may be specified at the beginning of
the \sphinxcode{\sphinxupquote{Returns}} section followed by a colon.

The \sphinxcode{\sphinxupquote{Returns}} section may span multiple lines and paragraphs.
Following lines should be indented to match the first line.

The \sphinxcode{\sphinxupquote{Returns}} section supports any reStructuredText formatting,
including literal blocks:

\begin{sphinxVerbatim}[commandchars=\\\{\}]
\PYG{p}{\PYGZob{}}
    \PYG{l+s+s1}{\PYGZsq{}}\PYG{l+s+s1}{param1}\PYG{l+s+s1}{\PYGZsq{}}\PYG{p}{:} \PYG{n}{param1}\PYG{p}{,}
    \PYG{l+s+s1}{\PYGZsq{}}\PYG{l+s+s1}{param2}\PYG{l+s+s1}{\PYGZsq{}}\PYG{p}{:} \PYG{n}{param2}
\PYG{p}{\PYGZcb{}}
\end{sphinxVerbatim}


\item[{Return type}] \leavevmode
bool

\end{description}\end{quote}

\begin{sphinxadmonition}{note}{Note:}
You asshole
\end{sphinxadmonition}

\begin{sphinxVerbatim}[commandchars=\\\{\}]
\PYG{g+go}{\PYGZgt{}\PYGZgt{}\PYGZgt{}\PYGZgt{}\PYGZgt{} pipo is log\PYGZus{}likelihood}
\PYG{g+go}{\PYGZgt{}\PYGZgt{}\PYGZgt{}\PYGZgt{}\PYGZgt{} i don\PYGZsq{}t like pipo}
\end{sphinxVerbatim}
\subsubsection*{Example}

Display text is okay, math is \(\sqrt{z + 1}\)

\begin{sphinxVerbatim}[commandchars=\\\{\}]
\PYG{g+gp}{\PYGZgt{}\PYGZgt{}\PYGZgt{} }\PYG{n}{numpy}\PYG{o}{.}\PYG{n}{buy}\PYG{p}{(}\PYG{l+m+mi}{22}\PYG{p}{)} \PYG{o}{=} \PYG{l+m+mi}{5}
\end{sphinxVerbatim}
\begin{equation*}
\begin{split}\sqrt{z}\end{split}
\end{equation*}
\end{fulllineitems}

\index{get\_stretch() (in module AIE\_sampling)@\spxentry{get\_stretch()}\spxextra{in module AIE\_sampling}}

\begin{fulllineitems}
\phantomsection\label{\detokenize{autogen:AIE_sampling.get_stretch}}\pysiglinewithargsret{\sphinxcode{\sphinxupquote{AIE\_sampling.}}\sphinxbfcode{\sphinxupquote{get\_stretch}}}{\emph{\DUrole{n}{a}}}{}
Generator of numbers distibuted as \(\frac{1}{\sqrt{z}} in [1/a,a]\)
Uses Inverse transform sampling

\end{fulllineitems}

\phantomsection\label{\detokenize{autogen:module-example_docstring}}\index{module@\spxentry{module}!example\_docstring@\spxentry{example\_docstring}}\index{example\_docstring@\spxentry{example\_docstring}!module@\spxentry{module}}
Example NumPy style docstrings.

This module demonstrates documentation as specified by the \sphinxhref{https://github.com/numpy/numpy/blob/master/doc/HOWTO\_DOCUMENT.rst.txt}{NumPy
Documentation HOWTO}%
\begin{footnote}[1]\sphinxAtStartFootnote
\sphinxnolinkurl{https://github.com/numpy/numpy/blob/master/doc/HOWTO\_DOCUMENT.rst.txt}
%
\end{footnote}. Docstrings may extend over multiple lines. Sections
are created with a section header followed by an underline of equal length.
\subsubsection*{Example}

Examples can be given using either the \sphinxcode{\sphinxupquote{Example}} or \sphinxcode{\sphinxupquote{Examples}}
sections. Sections support any reStructuredText formatting, including
literal blocks:

\begin{sphinxVerbatim}[commandchars=\\\{\}]
\PYGZdl{} python example\PYGZus{}numpy.py
\end{sphinxVerbatim}

Section breaks are created with two blank lines. Section breaks are also
implicitly created anytime a new section starts. Section bodies \sphinxstyleemphasis{may} be
indented:
\subsubsection*{Notes}

This is an example of an indented section. It’s like any other section,
but the body is indented to help it stand out from surrounding text.

If a section is indented, then a section break is created by
resuming unindented text.
\index{module\_level\_variable1 (in module example\_docstring)@\spxentry{module\_level\_variable1}\spxextra{in module example\_docstring}}

\begin{fulllineitems}
\phantomsection\label{\detokenize{autogen:example_docstring.module_level_variable1}}\pysigline{\sphinxcode{\sphinxupquote{example\_docstring.}}\sphinxbfcode{\sphinxupquote{module\_level\_variable1}}}
Module level variables may be documented in either the \sphinxcode{\sphinxupquote{Attributes}}
section of the module docstring, or in an inline docstring immediately
following the variable.

Either form is acceptable, but the two should not be mixed. Choose
one convention to document module level variables and be consistent
with it.
\begin{quote}\begin{description}
\item[{Type}] \leavevmode
int

\end{description}\end{quote}

\end{fulllineitems}

\index{ExampleClass (class in example\_docstring)@\spxentry{ExampleClass}\spxextra{class in example\_docstring}}

\begin{fulllineitems}
\phantomsection\label{\detokenize{autogen:example_docstring.ExampleClass}}\pysiglinewithargsret{\sphinxbfcode{\sphinxupquote{class }}\sphinxcode{\sphinxupquote{example\_docstring.}}\sphinxbfcode{\sphinxupquote{ExampleClass}}}{\emph{\DUrole{n}{param1}}, \emph{\DUrole{n}{param2}}, \emph{\DUrole{n}{param3}}}{}
The summary line for a class docstring should fit on one line.

If the class has public attributes, they may be documented here
in an \sphinxcode{\sphinxupquote{Attributes}} section and follow the same formatting as a
function’s \sphinxcode{\sphinxupquote{Args}} section. Alternatively, attributes may be documented
inline with the attribute’s declaration (see \_\_init\_\_ method below).

Properties created with the \sphinxcode{\sphinxupquote{@property}} decorator should be documented
in the property’s getter method.
\index{attr1 (example\_docstring.ExampleClass attribute)@\spxentry{attr1}\spxextra{example\_docstring.ExampleClass attribute}}

\begin{fulllineitems}
\phantomsection\label{\detokenize{autogen:example_docstring.ExampleClass.attr1}}\pysigline{\sphinxbfcode{\sphinxupquote{attr1}}}
Description of \sphinxtitleref{attr1}.
\begin{quote}\begin{description}
\item[{Type}] \leavevmode
str

\end{description}\end{quote}

\end{fulllineitems}

\index{attr2 (example\_docstring.ExampleClass attribute)@\spxentry{attr2}\spxextra{example\_docstring.ExampleClass attribute}}

\begin{fulllineitems}
\phantomsection\label{\detokenize{autogen:example_docstring.ExampleClass.attr2}}\pysigline{\sphinxbfcode{\sphinxupquote{attr2}}}
Description of \sphinxtitleref{attr2}.
\begin{quote}\begin{description}
\item[{Type}] \leavevmode
\sphinxcode{\sphinxupquote{int}}, optional

\end{description}\end{quote}

\end{fulllineitems}

\index{\_\_init\_\_() (example\_docstring.ExampleClass method)@\spxentry{\_\_init\_\_()}\spxextra{example\_docstring.ExampleClass method}}

\begin{fulllineitems}
\phantomsection\label{\detokenize{autogen:example_docstring.ExampleClass.__init__}}\pysiglinewithargsret{\sphinxbfcode{\sphinxupquote{\_\_init\_\_}}}{\emph{\DUrole{n}{param1}}, \emph{\DUrole{n}{param2}}, \emph{\DUrole{n}{param3}}}{}
Example of docstring on the \_\_init\_\_ method.

The \_\_init\_\_ method may be documented in either the class level
docstring, or as a docstring on the \_\_init\_\_ method itself.

Either form is acceptable, but the two should not be mixed. Choose one
convention to document the \_\_init\_\_ method and be consistent with it.

\begin{sphinxadmonition}{note}{Note:}
Do not include the \sphinxtitleref{self} parameter in the \sphinxcode{\sphinxupquote{Parameters}} section.
\end{sphinxadmonition}
\begin{quote}\begin{description}
\item[{Parameters}] \leavevmode\begin{itemize}
\item {} 
\sphinxstyleliteralstrong{\sphinxupquote{param1}} (\sphinxstyleliteralemphasis{\sphinxupquote{str}}) – Description of \sphinxtitleref{param1}.

\item {} 
\sphinxstyleliteralstrong{\sphinxupquote{param2}} (\sphinxcode{\sphinxupquote{list}} of \sphinxcode{\sphinxupquote{str}}) – Description of \sphinxtitleref{param2}. Multiple
lines are supported.

\item {} 
\sphinxstyleliteralstrong{\sphinxupquote{param3}} (\sphinxcode{\sphinxupquote{int}}, optional) – Description of \sphinxtitleref{param3}.

\end{itemize}

\end{description}\end{quote}

\end{fulllineitems}

\index{attr3 (example\_docstring.ExampleClass attribute)@\spxentry{attr3}\spxextra{example\_docstring.ExampleClass attribute}}

\begin{fulllineitems}
\phantomsection\label{\detokenize{autogen:example_docstring.ExampleClass.attr3}}\pysigline{\sphinxbfcode{\sphinxupquote{attr3}}}
Doc comment \sphinxstyleemphasis{inline} with attribute

\end{fulllineitems}

\index{attr4 (example\_docstring.ExampleClass attribute)@\spxentry{attr4}\spxextra{example\_docstring.ExampleClass attribute}}

\begin{fulllineitems}
\phantomsection\label{\detokenize{autogen:example_docstring.ExampleClass.attr4}}\pysigline{\sphinxbfcode{\sphinxupquote{attr4}}}
Doc comment \sphinxstyleemphasis{before} attribute, with type specified
\begin{quote}\begin{description}
\item[{Type}] \leavevmode
list of str

\end{description}\end{quote}

\end{fulllineitems}

\index{attr5 (example\_docstring.ExampleClass attribute)@\spxentry{attr5}\spxextra{example\_docstring.ExampleClass attribute}}

\begin{fulllineitems}
\phantomsection\label{\detokenize{autogen:example_docstring.ExampleClass.attr5}}\pysigline{\sphinxbfcode{\sphinxupquote{attr5}}}
Docstring \sphinxstyleemphasis{after} attribute, with type specified.
\begin{quote}\begin{description}
\item[{Type}] \leavevmode
str

\end{description}\end{quote}

\end{fulllineitems}

\index{example\_method() (example\_docstring.ExampleClass method)@\spxentry{example\_method()}\spxextra{example\_docstring.ExampleClass method}}

\begin{fulllineitems}
\phantomsection\label{\detokenize{autogen:example_docstring.ExampleClass.example_method}}\pysiglinewithargsret{\sphinxbfcode{\sphinxupquote{example\_method}}}{\emph{\DUrole{n}{param1}}, \emph{\DUrole{n}{param2}}}{}
Class methods are similar to regular functions.

\begin{sphinxadmonition}{note}{Note:}
Do not include the \sphinxtitleref{self} parameter in the \sphinxcode{\sphinxupquote{Parameters}} section.
\end{sphinxadmonition}
\begin{quote}\begin{description}
\item[{Parameters}] \leavevmode\begin{itemize}
\item {} 
\sphinxstyleliteralstrong{\sphinxupquote{param1}} – The first parameter.

\item {} 
\sphinxstyleliteralstrong{\sphinxupquote{param2}} – The second parameter.

\end{itemize}

\item[{Returns}] \leavevmode
True if successful, False otherwise.

\item[{Return type}] \leavevmode
bool

\end{description}\end{quote}

\end{fulllineitems}

\index{readonly\_property() (example\_docstring.ExampleClass property)@\spxentry{readonly\_property()}\spxextra{example\_docstring.ExampleClass property}}

\begin{fulllineitems}
\phantomsection\label{\detokenize{autogen:example_docstring.ExampleClass.readonly_property}}\pysigline{\sphinxbfcode{\sphinxupquote{property }}\sphinxbfcode{\sphinxupquote{readonly\_property}}}
Properties should be documented in their getter method.
\begin{quote}\begin{description}
\item[{Type}] \leavevmode
str

\end{description}\end{quote}

\end{fulllineitems}

\index{readwrite\_property() (example\_docstring.ExampleClass property)@\spxentry{readwrite\_property()}\spxextra{example\_docstring.ExampleClass property}}

\begin{fulllineitems}
\phantomsection\label{\detokenize{autogen:example_docstring.ExampleClass.readwrite_property}}\pysigline{\sphinxbfcode{\sphinxupquote{property }}\sphinxbfcode{\sphinxupquote{readwrite\_property}}}
Properties with both a getter and setter
should only be documented in their getter method.

If the setter method contains notable behavior, it should be
mentioned here.
\begin{quote}\begin{description}
\item[{Type}] \leavevmode
\sphinxcode{\sphinxupquote{list}} of \sphinxcode{\sphinxupquote{str}}

\end{description}\end{quote}

\end{fulllineitems}


\end{fulllineitems}

\index{ExampleError@\spxentry{ExampleError}}

\begin{fulllineitems}
\phantomsection\label{\detokenize{autogen:example_docstring.ExampleError}}\pysiglinewithargsret{\sphinxbfcode{\sphinxupquote{exception }}\sphinxcode{\sphinxupquote{example\_docstring.}}\sphinxbfcode{\sphinxupquote{ExampleError}}}{\emph{\DUrole{n}{msg}}, \emph{\DUrole{n}{code}}}{}
Exceptions are documented in the same way as classes.

The \_\_init\_\_ method may be documented in either the class level
docstring, or as a docstring on the \_\_init\_\_ method itself.

Either form is acceptable, but the two should not be mixed. Choose one
convention to document the \_\_init\_\_ method and be consistent with it.

\begin{sphinxadmonition}{note}{Note:}
Do not include the \sphinxtitleref{self} parameter in the \sphinxcode{\sphinxupquote{Parameters}} section.
\end{sphinxadmonition}
\begin{quote}\begin{description}
\item[{Parameters}] \leavevmode\begin{itemize}
\item {} 
\sphinxstyleliteralstrong{\sphinxupquote{msg}} (\sphinxstyleliteralemphasis{\sphinxupquote{str}}) – Human readable string describing the exception.

\item {} 
\sphinxstyleliteralstrong{\sphinxupquote{code}} (\sphinxcode{\sphinxupquote{int}}, optional) – Numeric error code.

\end{itemize}

\end{description}\end{quote}
\index{msg (example\_docstring.ExampleError attribute)@\spxentry{msg}\spxextra{example\_docstring.ExampleError attribute}}

\begin{fulllineitems}
\phantomsection\label{\detokenize{autogen:example_docstring.ExampleError.msg}}\pysigline{\sphinxbfcode{\sphinxupquote{msg}}}
Human readable string describing the exception.
\begin{quote}\begin{description}
\item[{Type}] \leavevmode
str

\end{description}\end{quote}

\end{fulllineitems}

\index{code (example\_docstring.ExampleError attribute)@\spxentry{code}\spxextra{example\_docstring.ExampleError attribute}}

\begin{fulllineitems}
\phantomsection\label{\detokenize{autogen:example_docstring.ExampleError.code}}\pysigline{\sphinxbfcode{\sphinxupquote{code}}}
Numeric error code.
\begin{quote}\begin{description}
\item[{Type}] \leavevmode
int

\end{description}\end{quote}

\end{fulllineitems}

\index{\_\_init\_\_() (example\_docstring.ExampleError method)@\spxentry{\_\_init\_\_()}\spxextra{example\_docstring.ExampleError method}}

\begin{fulllineitems}
\phantomsection\label{\detokenize{autogen:example_docstring.ExampleError.__init__}}\pysiglinewithargsret{\sphinxbfcode{\sphinxupquote{\_\_init\_\_}}}{\emph{\DUrole{n}{msg}}, \emph{\DUrole{n}{code}}}{}
Initialize self.  See help(type(self)) for accurate signature.

\end{fulllineitems}


\end{fulllineitems}

\index{example\_generator() (in module example\_docstring)@\spxentry{example\_generator()}\spxextra{in module example\_docstring}}

\begin{fulllineitems}
\phantomsection\label{\detokenize{autogen:example_docstring.example_generator}}\pysiglinewithargsret{\sphinxcode{\sphinxupquote{example\_docstring.}}\sphinxbfcode{\sphinxupquote{example\_generator}}}{\emph{\DUrole{n}{n}}}{}
Generators have a \sphinxcode{\sphinxupquote{Yields}} section instead of a \sphinxcode{\sphinxupquote{Returns}} section.
\begin{quote}\begin{description}
\item[{Parameters}] \leavevmode
\sphinxstyleliteralstrong{\sphinxupquote{n}} (\sphinxstyleliteralemphasis{\sphinxupquote{int}}) – The upper limit of the range to generate, from 0 to \sphinxtitleref{n} \sphinxhyphen{} 1.

\item[{Yields}] \leavevmode
\sphinxstyleemphasis{int} – The next number in the range of 0 to \sphinxtitleref{n} \sphinxhyphen{} 1.

\end{description}\end{quote}
\subsubsection*{Examples}

Examples should be written in doctest format, and should illustrate how
to use the function.

\begin{sphinxVerbatim}[commandchars=\\\{\}]
\PYG{g+gp}{\PYGZgt{}\PYGZgt{}\PYGZgt{} }\PYG{n+nb}{print}\PYG{p}{(}\PYG{p}{[}\PYG{n}{i} \PYG{k}{for} \PYG{n}{i} \PYG{o+ow}{in} \PYG{n}{example\PYGZus{}generator}\PYG{p}{(}\PYG{l+m+mi}{4}\PYG{p}{)}\PYG{p}{]}\PYG{p}{)}
\PYG{g+go}{[0, 1, 2, 3]}
\end{sphinxVerbatim}

\end{fulllineitems}

\index{function\_with\_pep484\_type\_annotations() (in module example\_docstring)@\spxentry{function\_with\_pep484\_type\_annotations()}\spxextra{in module example\_docstring}}

\begin{fulllineitems}
\phantomsection\label{\detokenize{autogen:example_docstring.function_with_pep484_type_annotations}}\pysiglinewithargsret{\sphinxcode{\sphinxupquote{example\_docstring.}}\sphinxbfcode{\sphinxupquote{function\_with\_pep484\_type\_annotations}}}{\emph{\DUrole{n}{param1}\DUrole{p}{:} \DUrole{n}{int}}, \emph{\DUrole{n}{param2}\DUrole{p}{:} \DUrole{n}{str}}}{{ $\rightarrow$ bool}}
Example function with PEP 484 type annotations.

The return type must be duplicated in the docstring to comply
with the NumPy docstring style.
\begin{quote}\begin{description}
\item[{Parameters}] \leavevmode\begin{itemize}
\item {} 
\sphinxstyleliteralstrong{\sphinxupquote{param1}} – The first parameter.

\item {} 
\sphinxstyleliteralstrong{\sphinxupquote{param2}} – The second parameter.

\end{itemize}

\item[{Returns}] \leavevmode
True if successful, False otherwise.

\item[{Return type}] \leavevmode
bool

\end{description}\end{quote}

\end{fulllineitems}

\index{function\_with\_types\_in\_docstring() (in module example\_docstring)@\spxentry{function\_with\_types\_in\_docstring()}\spxextra{in module example\_docstring}}

\begin{fulllineitems}
\phantomsection\label{\detokenize{autogen:example_docstring.function_with_types_in_docstring}}\pysiglinewithargsret{\sphinxcode{\sphinxupquote{example\_docstring.}}\sphinxbfcode{\sphinxupquote{function\_with\_types\_in\_docstring}}}{\emph{\DUrole{n}{param1}}, \emph{\DUrole{n}{param2}}}{}
Example function with types documented in the docstring.

{\color{red}\bfseries{}`PEP 484`\_} type annotations are supported. If attribute, parameter, and
return types are annotated according to {\color{red}\bfseries{}`PEP 484`\_}, they do not need to be
included in the docstring:
\begin{quote}\begin{description}
\item[{Parameters}] \leavevmode\begin{itemize}
\item {} 
\sphinxstyleliteralstrong{\sphinxupquote{param1}} (\sphinxstyleliteralemphasis{\sphinxupquote{int}}) – The first parameter.

\item {} 
\sphinxstyleliteralstrong{\sphinxupquote{param2}} (\sphinxstyleliteralemphasis{\sphinxupquote{str}}) – The second parameter.

\end{itemize}

\item[{Returns}] \leavevmode
\begin{itemize}
\item {} 
\sphinxstyleemphasis{bool} – True if successful, False otherwise.

\item {} 
\sphinxstyleemphasis{.. \_PEP 484} – \sphinxurl{https://www.python.org/dev/peps/pep-0484/}

\end{itemize}


\end{description}\end{quote}

\end{fulllineitems}

\index{module\_level\_function() (in module example\_docstring)@\spxentry{module\_level\_function()}\spxextra{in module example\_docstring}}

\begin{fulllineitems}
\phantomsection\label{\detokenize{autogen:example_docstring.module_level_function}}\pysiglinewithargsret{\sphinxcode{\sphinxupquote{example\_docstring.}}\sphinxbfcode{\sphinxupquote{module\_level\_function}}}{\emph{\DUrole{n}{param1}}, \emph{\DUrole{n}{param2}\DUrole{o}{=}\DUrole{default_value}{None}}, \emph{\DUrole{o}{*}\DUrole{n}{args}}, \emph{\DUrole{o}{**}\DUrole{n}{kwargs}}}{}
This is an example of a module level function.

Function parameters should be documented in the \sphinxcode{\sphinxupquote{Parameters}} section.
The name of each parameter is required. The type and description of each
parameter is optional, but should be included if not obvious.

If *args or **kwargs are accepted,
they should be listed as \sphinxcode{\sphinxupquote{*args}} and \sphinxcode{\sphinxupquote{**kwargs}}.

The format for a parameter is:

\begin{sphinxVerbatim}[commandchars=\\\{\}]
\PYG{n}{name} \PYG{p}{:} \PYG{n+nb}{type}
    \PYG{n}{description}

    \PYG{n}{The} \PYG{n}{description} \PYG{n}{may} \PYG{n}{span} \PYG{n}{multiple} \PYG{n}{lines}\PYG{o}{.} \PYG{n}{Following} \PYG{n}{lines}
    \PYG{n}{should} \PYG{n}{be} \PYG{n}{indented} \PYG{n}{to} \PYG{n}{match} \PYG{n}{the} \PYG{n}{first} \PYG{n}{line} \PYG{n}{of} \PYG{n}{the} \PYG{n}{description}\PYG{o}{.}
    \PYG{n}{The} \PYG{l+s+s2}{\PYGZdq{}}\PYG{l+s+s2}{: type}\PYG{l+s+s2}{\PYGZdq{}} \PYG{o+ow}{is} \PYG{n}{optional}\PYG{o}{.}

    \PYG{n}{Multiple} \PYG{n}{paragraphs} \PYG{n}{are} \PYG{n}{supported} \PYG{o+ow}{in} \PYG{n}{parameter}
    \PYG{n}{descriptions}\PYG{o}{.}
\end{sphinxVerbatim}
\begin{quote}\begin{description}
\item[{Parameters}] \leavevmode\begin{itemize}
\item {} 
\sphinxstyleliteralstrong{\sphinxupquote{param1}} (\sphinxstyleliteralemphasis{\sphinxupquote{int}}) – The first parameter.

\item {} 
\sphinxstyleliteralstrong{\sphinxupquote{param2}} (\sphinxcode{\sphinxupquote{str}}, optional) – The second parameter.

\item {} 
\sphinxstyleliteralstrong{\sphinxupquote{*args}} – Variable length argument list.

\item {} 
\sphinxstyleliteralstrong{\sphinxupquote{**kwargs}} – Arbitrary keyword arguments.

\end{itemize}

\item[{Returns}] \leavevmode

True if successful, False otherwise.

The return type is not optional. The \sphinxcode{\sphinxupquote{Returns}} section may span
multiple lines and paragraphs. Following lines should be indented to
match the first line of the description.

The \sphinxcode{\sphinxupquote{Returns}} section supports any reStructuredText formatting,
including literal blocks:

\begin{sphinxVerbatim}[commandchars=\\\{\}]
\PYG{p}{\PYGZob{}}
    \PYG{l+s+s1}{\PYGZsq{}}\PYG{l+s+s1}{param1}\PYG{l+s+s1}{\PYGZsq{}}\PYG{p}{:} \PYG{n}{param1}\PYG{p}{,}
    \PYG{l+s+s1}{\PYGZsq{}}\PYG{l+s+s1}{param2}\PYG{l+s+s1}{\PYGZsq{}}\PYG{p}{:} \PYG{n}{param2}
\PYG{p}{\PYGZcb{}}
\end{sphinxVerbatim}


\item[{Return type}] \leavevmode
bool

\item[{Raises}] \leavevmode\begin{itemize}
\item {} 
\sphinxstyleliteralstrong{\sphinxupquote{AttributeError}} – The \sphinxcode{\sphinxupquote{Raises}} section is a list of all exceptions
    that are relevant to the interface.

\item {} 
\sphinxstyleliteralstrong{\sphinxupquote{ValueError}} – If \sphinxtitleref{param2} is equal to \sphinxtitleref{param1}.

\end{itemize}

\end{description}\end{quote}

\end{fulllineitems}

\index{module\_level\_variable2 (in module example\_docstring)@\spxentry{module\_level\_variable2}\spxextra{in module example\_docstring}}

\begin{fulllineitems}
\phantomsection\label{\detokenize{autogen:example_docstring.module_level_variable2}}\pysigline{\sphinxcode{\sphinxupquote{example\_docstring.}}\sphinxbfcode{\sphinxupquote{module\_level\_variable2}}\sphinxbfcode{\sphinxupquote{ = 98765}}}
Module level variable documented inline.

The docstring may span multiple lines. The type may optionally be specified
on the first line, separated by a colon.
\begin{quote}\begin{description}
\item[{Type}] \leavevmode
int

\end{description}\end{quote}

\end{fulllineitems}


\end{description}


\chapter{Indices and tables}
\label{\detokenize{index:indices-and-tables}}\begin{itemize}
\item {} 
\DUrole{xref,std,std-ref}{genindex}

\item {} 
\DUrole{xref,std,std-ref}{modindex}

\item {} 
\DUrole{xref,std,std-ref}{search}

\end{itemize}


\renewcommand{\indexname}{Python Module Index}
\begin{sphinxtheindex}
\let\bigletter\sphinxstyleindexlettergroup
\bigletter{a}
\item\relax\sphinxstyleindexentry{AIE\_sampling}\sphinxstyleindexpageref{autogen:\detokenize{module-AIE_sampling}}
\indexspace
\bigletter{e}
\item\relax\sphinxstyleindexentry{example\_docstring}\sphinxstyleindexpageref{autogen:\detokenize{module-example_docstring}}
\indexspace
\bigletter{m}
\item\relax\sphinxstyleindexentry{my\_NS}\sphinxstyleindexpageref{autogen:\detokenize{module-my_NS}}
\end{sphinxtheindex}

\renewcommand{\indexname}{Index}
\footnotesize\raggedright\printindex
\end{document}